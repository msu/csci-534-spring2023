\documentclass{article}
\usepackage{../../fasy-hw}

\title{Computational Geometry, Homework \hwnum}
\collab{n/a}

%% Instructor: update these macros:
\renewcommand{\hwnum}{6}
\date{due: LDOC}

%% Student: update this macro:
\author{\todo{Your Name Here}}

\begin{document}

\maketitle

This homework assignment should be
submitted as a single PDF file to D2L.

General homework expectations:
\begin{itemize}
    \item Homework should be typeset using LaTex.
    \item Answers should be in complete sentences and proofread so that they
        make sense without seeing the question.
    \item You will not plagiarize, nor will you share your written solutions
        with classmates. (But, discussing the questions is highly encouraged).
    \item List collaborators at the start of each question using the
        \texttt{collab} command.
    \item Put your answers where the \texttt{todo} command currently is (and
        remove the \texttt{todo} macro, but not the word \texttt{Answer}).
    \item If you are asked to come up with an algorithm, you are
        expected to give an efficient algorithm (brute-force solutions will not
        be accepted). With your algorithm, please provide the following:
        \begin{itemize}
            \item \emph{What}: A prose explanation of the problem and the algorithm,
                including a description of the input/output.  Be sure to state
                your GP assumptions.
            \item \emph{How}: Describe how the algorithm works clearly.
                Including pseudocode may helpful.
            \item \emph{How Fast}: Runtime, along with the derivation.  (Or, at
                the very least, a proof of termination using a decrementing
                function).  You only need to specify the space complexity if the
                problem asks for it.
           \item \emph{Why}: Brief justification of why the algorithm works.
               Often, this will include a statement of the loop invariant for each
               (outer-most) loop, or recursion invariant for each recursive function.
        \end{itemize}
\end{itemize}

{\bf David Mount's tips for writing up homework solutions}:
Remember that your description is intended to be read by a
human, not a compiler, so conciseness and clarity are preferred over technical
details.  Unless otherwise stated, you may use any results from class, or
results from any standard textbook on algorithms and data structures. Also, you
may use results from geometry that: (1) have been mentioned in class, (2) would
be known to someone who knows basic geometry or linear algebra, or (3) is
intuitively obvious. If you are unsure, please feel free to check with me.

Giving careful and rigorous proofs can be quite cumbersome in geometry, and so
you are encouraged to use intuition and give illustrations whenever appropriate.
Beware, however, that a poorly drawn figure can make certain erroneous
hypotheses appear to be ``obviously correct.''

Throughout the semester, unless otherwise stated, you may assume that input
objects are in general position. For example, you may assume that no two points
have the same x-coordinate, no three points are collinear, no four points are
co-circular. Also, unless otherwise stated, you may assume that any geometric
primitive involving a constant number of objects each of constant complexity can
be computed in $\Theta(1)$ time


{\bf Acknowledgement}: the homework problems were adapted from assignments of David
Millman, which, in turn, were adaptations of problems by David Mount and Carola
Wenk.

%%%%%%%%%%%%%%%%%%%%%%%%%%%%%%%%%%%%%%%%%%%%%%%%%%%%%%%%%%%%%%%%%%%%%%%%%%%%%%
\collab{\todo{}}
\nextprob{Redo}

Choose a HW problem that you were a little shaky about your first submission.
Redo your solution.  If feedback has not yet been given back for this problem,
ping Brittany for feedback (if you ask for one problem, detailed feedback for
that will be given within 24 hours).

\paragraph{Answer}

\todo{replace this TODO with your answer}

%%%%%%%%%%%%%%%%%%%%%%%%%%%%%%%%%%%%%%%%%%%%%%%%%%%%%%%%%%%%%%%%%%%%%%%%%%%%%%

%%%%%%%%%%%%%%%%%%%%%%%%%%%%%%%%%%%%%%%%%%%%%%%%%%%%%%%%%%%%%%%%%%%%%%%%%%%%%%
\collab{\todo{}}
\nextprob{Voronoi Diagrams}

The Voronoi diagram of a set of points in $\R^2$ is the
projection of the upper envelope of the dual lifted set of planes in $\R^3$.
What does the projection of the lower envelope correspond to? Similarly, what
does the projection of the upper convex hull of the points lifted to $\R^3$
correspond to?

You MAY (but do not need to) answer these questions by researching on the
internet. Cite the sources you were using and give an explanation in your own
words

\paragraph{Answer}

\todo{replace this TODO with your answer}

%%%%%%%%%%%%%%%%%%%%%%%%%%%%%%%%%%%%%%%%%%%%%%%%%%%%%%%%%%%%%%%%%%%%%%%%%%%%%%

%%%%%%%%%%%%%%%%%%%%%%%%%%%%%%%%%%%%%%%%%%%%%%%%%%%%%%%%%%%%%%%%%%%%%%%%%%%%%%
\collab{\todo{}}
\nextprob{Arrangements}

Consider an arrangement $\A$ of six lines $\ell_1, \ell_2, \ldots, \ell_6$ and
let $f$ be an arbitrary vertex, edge, or face of $\A$. Then $f$ has an
associated sign vector $(s_1, s_2, s_3, s_4, s_5, s_6)$, where for each $1 \le i
\le 6$:

$$
    s_i =
    \begin{cases}
        +1, & \text{if $f$ lies above $l_i$} \\
        0,  & \text{if $f$ lies on $l_i$} \\
        -1, & \text{if $f$ lies below $l_i$}
    \end{cases}
$$

\begin{enumerate}

    \item For each of the sign vectors below, give an arrangement of six lines
        that has a vertex, edge, or face with this sign vector. Label the lines
        and mark the vertex, edge, or face. Make the arrangement simple, if
        possible, or argue why the arrangement cannot be simple.
        \begin{enumerate}
            \item $(+1, +1, +1, +1, +1, +1)$
            \item $(+1, 0, 0, -1, -1, -1)$
            \item $(-1, 0, 0, -1, +1, -1)$
            \item $(+1, -1, -1, -1, -1, -1)$
        \end{enumerate}

        \paragraph{Answer}
        \todo{replace this TODO with your answer}


    \item Can one find a single arrangement of lines that contains a vertex,
        edge, or face for each of the four sign vectors? If you can, provide an
        example.  If you cannot, argue why not.

        \paragraph{Answer}
        \todo{replace this TODO with your answer}

\end{enumerate}

%%%%%%%%%%%%%%%%%%%%%%%%%%%%%%%%%%%%%%%%%%%%%%%%%%%%%%%%%%%%%%%%%%%%%%%%%%%%%%


%%%%%%%%%%%%%%%%%%%%%%%%%%%%%%%%%%%%%%%%%%%%%%%%%%%%%%%%%%%%%%%%%%%%%%%%%%%%%%
\collab{\todo{}}
\nextprob{Advice}

Suppose someone is interested in studying computational geometry, and they just
reached out to you for advice.  Where would
you direct them to start learning about the subject? What would you tell them
are some of the most important concepts that they should learn about? Write an
email to them to help them get started.
(Note: your responses here may inform future updates to the CompTaG onboarding
document for future research students. Any thoughts/ideas are appreciated here!)

\paragraph{Answer}
\todo{replace this TODO with your answer}

%%%%%%%%%%%%%%%%%%%%%%%%%%%%%%%%%%%%%%%%%%%%%%%%%%%%%%%%%%%%%%%%%%%%%%%%%%%%%%


%%%%%%%%%%%%%%%%%%%%%%%%%%%%%%%%%%%%%%%%%%%%%%%%%%%%%%%%%%%%%%%%%%%%%%%%%%%%%%
\collab{\todo{}}
\nextprob{Reflection}

Explain to me anything that you tried differently on this homework, focused on
for improvement of technical writing, or what you would especially like feedback
on.  It can be something simple like ``Before this class, I was not very aware
of what tense I was writing in.  For this homework, I tried to focus on writing
in present tense.'' or something very focused, such as ``I really focused on
improving my End-condition proofs for loop invariants.''  If you tried something
and struggled, let me know. Since everyone needs to improve (including me),
there should be something that you can write about here.  You can base it on
feedback I gave you, feedback I gave the class, feedback you received from
someone else, an example of technical writing that you really liked, or anything
else that motivates you to improve.

\paragraph{Answer}

\todo{replace this TODO with your answer}
%%%%%%%%%%%%%%%%%%%%%%%%%%%%%%%%%%%%%%%%%%%%%%%%%%%%%%%%%%%%%%%%%%%%%%%%%%%%%%

\end{document}
